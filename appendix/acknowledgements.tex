% !TEX TS-program = XeLaTeX
% !TEX encoding = UTF-8 Unicode

%%%%%%%%%%%%%%%%%%%%%%%%%%%%%%%%%%%%%%%%%%%%%%%%%%%%%%%%%%%%%%%%%%%%%
%
%  哈尔滨工程大学硕士论文 XeLaTeX 模版 —— 致谢文件 acknowledgements.tex
%
%  版本:1.0.0
%  最后更新:
%  修改者:Leo LiWenhui lwh@hrbeu.edu.cn
%  修订者:
%  编译环境1:Ubuntu 12.04 + TeXLive 2013/2014
%  编译环境2:Windows 7/8  + TeXLive 2013/2014
%
%%%%%%%%%%%%%%%%%%%%%%%%%%%%%%%%%%%%%%%%%%%%%%%%%%%%%%%%%%%%%%%%%%%%%

\appendix{致  谢}

学位论文中不得书写与论文工作无关的人和事,对导师的致谢要实事求是。

一同工作的同志对本研究所做的贡献应在论文中做明确的说明并表示谢意。

这部分内容不可省略。

在这里,向所有协助测试的同学、朋友表示感谢。
